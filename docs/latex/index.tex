Version\+: 1.\+2.\+0~\newline
 Release date\+: 2020-\/09-\/11~\newline
 \href{https://travis-ci.org/pololu/zumo-32u4-arduino-library}{\tt }~\newline
 \href{http://www.pololu.com/}{\tt www.\+pololu.\+com}

This is a C++ library for the Arduino I\+DE that helps access the on-\/board hardware of the Zumo 32\+U4 robot.

The Zumo 32\+U4 robot consists of the Zumo chassis, the Zumo 32\+U4 Main Board, and the Zumo 32\+U4 Front Sensor Array. It has an integrated A\+VR A\+Tmega32\+U4 microcontroller, motor drivers, encoders, proximity sensors, line sensors, inertial sensors, buzzer, four buttons, L\+CD connector. The user\textquotesingle{}s guide for the Zumo 32\+U4 robot is here\+:

\href{https://www.pololu.com/docs/0J63}{\tt https\+://www.\+pololu.\+com/docs/0\+J63}

Please note that this library does N\+OT work with the Zumo Shield for Arduino, which is a very different product. The Zumo Shield does not have an integrated microcontroller, so it must be connected to an Arduino-\/sized board to run. If you have the Zumo Shield, then you should not use this library and instead refer to the Zumo Shield documentation \href{https://www.pololu.com/docs/0J57}{\tt here}.

If you are using version 1.\+6.\+2 or later of the \href{http://www.arduino.cc/en/Main/Software}{\tt Arduino software (I\+DE)}, you can use the Library Manager to install this library\+:


\begin{DoxyEnumerate}
\item In the Arduino I\+DE, open the \char`\"{}\+Sketch\char`\"{} menu, select \char`\"{}\+Include Library\char`\"{}, then \char`\"{}\+Manage Libraries...\char`\"{}.
\item Search for \char`\"{}\+Zumo32\+U4\char`\"{}.
\item Click the Zumo32\+U4 entry in the list.
\item Click \char`\"{}\+Install\char`\"{}.
\end{DoxyEnumerate}

If this does not work, you can manually install the library\+:


\begin{DoxyEnumerate}
\item Download the \href{https://github.com/pololu/zumo-32u4-arduino-library}{\tt latest release archive from Git\+Hub} and decompress it.
\item Rename the folder \char`\"{}zumo-\/32u4-\/arduino-\/library-\/master\char`\"{} to \char`\"{}\+Zumo32\+U4\char`\"{}.
\item Move the \char`\"{}\+Zumo32\+U4\char`\"{} folder into the \char`\"{}libraries\char`\"{} directory inside your Arduino sketchbook directory. You can view your sketchbook location by opening the \char`\"{}\+File\char`\"{} menu and selecting \char`\"{}\+Preferences\char`\"{} in the Arduino I\+DE. If there is not already a \char`\"{}libraries\char`\"{} folder in that location, you should make the folder yourself.
\item After installing the library, restart the Arduino I\+DE.
\end{DoxyEnumerate}

Several example sketches are available that show how to use the library. You can access them from the Arduino I\+DE by opening the \char`\"{}\+File\char`\"{} menu, selecting \char`\"{}\+Examples\char`\"{}, and then selecting \char`\"{}\+Zumo32\+U4\char`\"{}. If you cannot find these examples, the library was probably installed incorrectly and you should retry the installation instructions above.

The main classes and functions provided by the library are listed below\+:


\begin{DoxyItemize}
\item \hyperlink{class_zumo32_u4_button_a}{Zumo32\+U4\+ButtonA}
\item \hyperlink{class_zumo32_u4_button_b}{Zumo32\+U4\+ButtonB}
\item \hyperlink{class_zumo32_u4_button_c}{Zumo32\+U4\+ButtonC}
\item \hyperlink{class_zumo32_u4_buzzer}{Zumo32\+U4\+Buzzer}
\item \hyperlink{class_zumo32_u4_encoders}{Zumo32\+U4\+Encoders}
\item \hyperlink{class_zumo32_u4_i_m_u}{Zumo32\+U4\+I\+MU}
\item \hyperlink{class_zumo32_u4_i_r_pulses}{Zumo32\+U4\+I\+R\+Pulses}
\item \hyperlink{class_zumo32_u4_l_c_d}{Zumo32\+U4\+L\+CD}
\item \hyperlink{class_zumo32_u4_line_sensors}{Zumo32\+U4\+Line\+Sensors}
\item \hyperlink{class_zumo32_u4_motors}{Zumo32\+U4\+Motors}
\item \hyperlink{class_zumo32_u4_proximity_sensors}{Zumo32\+U4\+Proximity\+Sensors}
\item \hyperlink{_zumo32_u4_8h_ae6ec5117b26ffaaa1b81c8c8b34426e1}{led\+Red()}
\item \hyperlink{_zumo32_u4_8h_a22e68694b618fe149ed42d76e96597ca}{led\+Green()}
\item \hyperlink{_zumo32_u4_8h_a7528cb14b314ccde63c94049402d01c6}{led\+Yellow()}
\item \hyperlink{_zumo32_u4_8h_acf21c49681fe010784b631551ab921f3}{usb\+Power\+Present()}
\item \hyperlink{_zumo32_u4_8h_a9391e187045f8f5a48546b34b6c6db25}{read\+Battery\+Millivolts()}
\end{DoxyItemize}

This library also includes copies of several other Arduino libraries inside it which are used to help implement the classes and functions above.


\begin{DoxyItemize}
\item \href{https://github.com/pololu/fastgpio-arduino}{\tt Fast\+G\+P\+IO}
\item \href{https://github.com/pololu/pololu-buzzer-arduino}{\tt Pololu\+Buzzer}
\item \href{https://github.com/pololu/pololu-hd44780-arduino}{\tt Pololu\+H\+D44780}
\item \href{https://github.com/pololu/pushbutton-arduino}{\tt Pushbutton}
\item \href{https://github.com/pololu/qtr-sensors-arduino}{\tt Q\+T\+R\+Sensors}
\item \href{https://github.com/pololu/usb-pause-arduino}{\tt U\+S\+B\+Pause}
\end{DoxyItemize}

You can use these libraries in your sketch automatically without any extra installation steps and without needing to add any extra {\ttfamily \#include} lines to your sketch.

You should avoid adding extra {\ttfamily \#include} lines such as {\ttfamily \#include $<$\hyperlink{_pushbutton_8h}{Pushbutton.\+h}$>$} because then the Arduino I\+DE might try to use the standalone \hyperlink{class_pushbutton}{Pushbutton} library (if you previously installed it), and it would conflict with the copy of the \hyperlink{class_pushbutton}{Pushbutton} code included in this library. The only {\ttfamily \#include} lines needed to access all features of this library are\+:


\begin{DoxyCode}
\textcolor{preprocessor}{#include <Wire.h>}
\textcolor{preprocessor}{#include <\hyperlink{_zumo32_u4_8h}{Zumo32U4.h}>}
\end{DoxyCode}


For complete documentation, see \href{https://pololu.github.io/zumo-32u4-arduino-library}{\tt https\+://pololu.\+github.\+io/zumo-\/32u4-\/arduino-\/library}. If you are already on that page, then click on the links in the \char`\"{}\+Classes and functions\char`\"{} section above.


\begin{DoxyItemize}
\item 1.\+2.\+0 (2020-\/09-\/11)\+: Added a \hyperlink{class_zumo32_u4_i_m_u}{Zumo32\+U4\+I\+MU} class that abstracts some details of the inertial sensors and supports different I\+MU types. The examples have been updated to use this class.
\item 1.\+1.\+4 (2017-\/07-\/17)\+: Fixed a bug that caused errors from the right encoder to be reported as errors from the left encoder.
\item 1.\+1.\+3 (2016-\/10-\/12)\+: Fixed a bug that caused the buzzer\textquotesingle{}s {\ttfamily is\+Playing} method to malfunction sometimes when link time optimization is enabled. Also incorporated some minor fixes to the \hyperlink{class_q_t_r_sensors}{Q\+T\+R\+Sensors} and L\+S\+M303 libraries.
\item 1.\+1.\+2 (2016-\/03-\/14)\+: Updated the Demo example so it can compile in Arduino 1.\+6.\+7.
\item 1.\+1.\+1 (2015-\/09-\/01)\+: Moved the library out of the a-\/star repository into its own repository. Added Demo example.
\item 1.\+1.\+0 (2015-\/05-\/06)\+: Updated Fast\+G\+P\+IO to version 1.\+0.\+2. Fixed a bug in \hyperlink{class_zumo32_u4_proximity_sensors}{Zumo32\+U4\+Proximity\+Sensors} where the wrong array length was used. Added five demos\+: Rotation\+Resist, Face\+Uphill, Remote\+Control, Balancing, and Sumo\+Proximity\+Sensors.
\item 1.\+0.\+1 (2015-\/03-\/11)\+: Improve the Buttons example.
\item 1.\+0.\+0 (2015-\/03-\/05)\+: Original release. 
\end{DoxyItemize}